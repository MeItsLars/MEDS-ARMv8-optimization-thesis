\documentclass[11pt,a4paper]{report}

\usepackage{url}
\usepackage[utf8]{inputenc}
\usepackage{graphics}
\usepackage{graphicx}
\setlength{\marginparwidth}{2cm}
\usepackage{todonotes}
\usepackage{natbib}
\usepackage{enumitem}
\usepackage{xurl}
\usepackage{hyperref}
\hypersetup{
    colorlinks,
    citecolor=black,
    filecolor=black,
    linkcolor=black,
    urlcolor=black
}
\usepackage[a4paper]{geometry}

% Set the line spacing between paragraphs
\setlength{\parskip}{1em}
% Remove paragraph indentation
\setlength{\parindent}{0pt}

% URL LINE BREAKS
\expandafter\def\expandafter\UrlBreaks\expandafter{\UrlBreaks% save the current one
  \do\a\do\b\do\c\do\d\do\e\do\f\do\g\do\h\do\i\do\j%
  \do\k\do\l\do\m\do\n\do\o\do\p\do\q\do\r\do\s\do\t%
  \do\u\do\v\do\w\do\x\do\y\do\z\do\A\do\B\do\C\do\D%
  \do\E\do\F\do\G\do\H\do\I\do\J\do\K\do\L\do\M\do\N%
  \do\O\do\P\do\Q\do\R\do\S\do\T\do\U\do\V\do\W\do\X%
  \do\Y\do\Z\do\*\do\-\do\~\do\'\do\"\do\-}%

\makeatletter %otherwise geometry resets everything
\Gm@restore@org
\makeatother

\setlength{\itemsep}{0cm}
\setlength{\voffset}{0cm}
\setlength{\headheight}{0cm}
\setlength{\topmargin}{0cm}

\graphicspath{{imgs/}}

\begin{document}
\begin{titlepage}
\begin{center}
\textsc{\LARGE Master Thesis\\Computing Science}\\[1.5cm]
\includegraphics[height=100pt]{logo}

\vspace{0.4cm}
\textsc{\Large Radboud University}\\[1cm]
\hrule
\vspace{0.4cm}
\textbf{\huge Optimizing MEDS implementation for ARMv8}\\[0.4cm]
\hrule
\vspace{2cm}
\begin{minipage}[t]{0.45\textwidth}
\begin{flushleft} \large
\textit{Author:}\\
Lars Jeurissen\\
s1022856\\
\texttt{lars.jeurissen@ru.nl}
\end{flushleft}
\end{minipage}
\begin{minipage}[t]{0.45\textwidth}
\begin{flushright} \large
\textit{First supervisor/assessor:}\\
prof. dr. Peter Schwabe\\
\texttt{p.schwabe@cs.ru.nl}\\[1.3cm]
\textit{Second supervisor:}\\
dr. Simona Samardjiska\\
\texttt{simonas@cs.ru.nl}
\end{flushright}
\end{minipage}
\vfill
{\large \today}
\end{center}
\end{titlepage}

\renewcommand{\abstractname}{Acknowledgements}
\begin{abstract}
    
    \todo[inline]{Include acknowledgements.}

\end{abstract}

\renewcommand{\abstractname}{Abstract}
\begin{abstract}
    As the threat of quantum computers increases, the need for post-quantum cryptography becomes more pressing. The recent NIST competition on post-quantum signature schemes has led to the creation of MEDS, a signature scheme based on the Matrix Code Equivalence (MCE) problem. In this work, we optimize the existing MEDS implementation for the ARMv8 CPU architecture using parameter-specific optimizations and NEON SIMD instructions.\\
    We explore two approaches: a low-level approach that optimizes the individual operations of the scheme and a high-level approach that parallelizes over the main commitment loop. The low-level approach gives the best results, with a speedup of 3.1x for key generation, 3.8x for signing, and 4.0x for verification, in NIST category 3.\\
    Additionally, we suggest an optimization to the hashing structure used in MEDS which increases the speedup for signing and verification to 4.0x and 4.2x, respectively, in NIST category 3.
\end{abstract}

\tableofcontents
\newpage

\chapter{Introduction}
\label{ch:introduction}
As the research on quantum computers progresses, we are getting increasingly closer to the point where quantum computers will be able to utilize algorithms such as Shor's algorithm and Grover's algorithm to break a lot of symmetric and asymmetric cryptographic schemes. As the majority of the internet's security relies on these cryptographic schemes, the consequences of this are severe. Without proper measurements, quantum computers will cause the absolute collapse of the present public key algorithms that are considered secure \cite{mavroeidis2018impact}, wich would have devestating consequences for the security of the internet.

The solution to this problem lies in the development of cryptographic schemes that are secure against quantum computers. Such algorithms have been around for a long time, but this area of research has experienced a boost in attention ever since the National Institute of Standards and Technology (NIST) started the post-quantum cryptography (PQC) standardization process in 2017 \cite{nist2017pqc}. The goal of this process is to standardize cryptographic schemes that are secure against quantum computers.

In 2022, NIST announced the set of selected PQC algorithms. Unfortunately, this set did not include any algorithms for digital signatures. To address this, NIST announced a second competition in the PQC standardization process, which aims to find a set of secure digital signature schemes. One of the candidates in this competition is Matrix Equivalence Digital Signature (MEDS) \cite{chou2023take}. MEDS is a code-based digital signature scheme based on the notion of Matrix Code Equivalence.

Although the MEDS scheme is actively being optimized in terms of key and signature sizes, the performance of the scheme is still lacking. The reported signature verification times are in the order of hundreds of milliseconds. \todo[inline]{Comparison with other similar schemes? Or this go in MEDS section?}

In this thesis, we aim to optimize the speed of the MEDS implementation. We will look at the general performance of the MEDS implementation, but we will mostly focus on the ARMv8 architecture. This architecture is used in a wide range of devices, including many mobile devices and tablets, Internet of Things (IoT) devices, and Apple M1/M2 chips. By optimizing the MEDS implementation for ARMv8, we aim to improve the MEDS performance for these devices, as well as obtain a better understanding of the performance of MEDS.

We investigate the following research questions in this thesis (see also Chapter \ref{ch:researchobjectives}):
\begin{itemize}
    \item What are the bottlenecks in the MEDS implementation?
    \item How can we improve the general performance of the MEDS implementation?
    \item How can we optimize the MEDS implementation for ARMv8?
    \item How does the optimized MEDS implementation compare to (optimized) implementations of other digital signature schemes?
\end{itemize}

The underlying background of this thesis is explained in Chapter \ref{ch:background}.
\todo[inline]{Add more about the structure of the thesis.}

\chapter{Background}
\label{ch:background}
\todo[inline]{Explain Nist PQC? Or introduction?}
\todo[inline]{Explain Signature Schemes}
\todo[inline]{Explain MEDS}
MEDS \cite{chou2023take}
\todo[inline]{Explain ARMv8 and NEON}
\todo[inline]{Explain profiling techniques}

\chapter{Research Objectives}
\label{ch:researchobjectives}
\todo[inline]{Explain MEDS speedup goals}
\todo[inline]{Explain SIMD speedup}
\todo[inline]{Explain ARM speedup goals (IoT, mobile, etc.)}

\chapter{Methodology}
\label{ch:methodology}
\todo[inline]{Give MEDS profiling results on ARM}
\todo[inline]{For every optimized function, give a minimum cycle bound and a reasoning. Explain the techniques used to optimize the functions.}
\todo[inline]{Check; should I include the performance results in this section or in the results section? What about intermediate results?}

\chapter{Discussion}
\label{ch:discussion}
\todo[inline]{Discuss the results.}

\chapter{Future work}
\label{ch:futurework}
\todo[inline]{Discuss future work.}
- Check if ALTEQ optimization can be applied to MEDS.
- Create a non-constant time implementation for gaussian elimination (can be used in verification).
- Create a parallel implementation for gaussian elimination.
- Check if it is possible to parallelize over the entire challenge \& response loop.

\chapter{Conclusion}
\label{ch:conclusion}
\todo[inline]{Conclude the thesis.}

\bibliographystyle{plain}
\bibliography{bibliofile}

\appendix
\chapter{MEDS Operations}
Epic operations.

\newpage
\section{Key Generation}
\begin{algorithm}
\caption{MEDS.KeyGen()}\label{alg:medskeygen}
\hspace*{\algorithmicindent} \textbf{Input:} -\\
\hspace*{\algorithmicindent} \textbf{Output:} public key $\textbf{pk} \in \mathcal{B}^{\ell_\textbf{pk}}$, secret key $\textbf{sk} \in \mathcal{B}^{\ell_\textbf{sk}}$
\begin{algorithmic}[1]
% Generate a random secret seed
\State $\delta \in \mathcal{B}^{\ell_\text{sec\_seed}} \gets \text{Randombytes}(\ell_\text{sec\_seed})$
% Generate random secret and public seed from the previously generated secret seed
\State $\sigma_{\textbf{G}_0} \in \mathcal{B}^{\ell_\text{pub\_seed}}, \sigma \in \mathcal{B}^{\ell_\text{sec\_seed}} \gets \text{XOF}(\delta, \ell_\text{pub\_seed}, \ell_\text{sec\_seed})$
% Generate a random matrix G_0 from the public seed
\State $\textbf{G}_0 \in \mathds{F}_q^{k \times mn} \gets \text{ExpandSysMat}(\sigma_{\textbf{G}_0})$
% Generate G_i for every s
\ForAll{$i \in \{1, \ldots, s - 1\}$}
    % Generate two new seeds from the current state of the secret seed and replace the current secret seed
    \State $\sigma_{\textbf{T}_i}, \sigma \in \mathcal{B}^{\ell_\text{sec\_seed}} \gets \text{XOF}(\sigma, \ell_\text{sec\_seed}, \ell_\text{sec\_seed})$
    % Generate a random invertible matrix T_i
    \State $\textbf{T}_i \in \text{GL}_k(q) \gets \text{ExpandInvMat}(\sigma_{\textbf{T}_i}, k)$
    % Compute G_0' = T_i * G_0
    \State $\textbf{G}_0' \in \mathds{F}_q^{k \times mn} \gets \textbf{T}_i \textbf{G}_0$
    % Solve system of equations to obtain A and B
    \State $\check{\textbf{A}}_i \in \mathds{F}_q^{m \times m} \cup \{\bot\}, \check{\textbf{B}}_i \in \mathds{F}_q^{n \times n} \cup \{\bot\} \gets \text{Solve}(\textbf{G}_0')$
    % Retry if there was no solution
    \If{$(\check{\textbf{A}}_i = \bot \textbf{ and } \check{\textbf{B}}_i = \bot) \textbf{ or } \check{\textbf{A}}_i \notin \text{GL}_m(q) \textbf{ or } \check{\textbf{B}}_i \notin \text{GL}_n(q)$}
        \State \textbf{goto} line 5
    \EndIf
    % Get A_i, A_i^-1, B_i, and B_i^-1 from the solution
    \State $\textbf{A}_i, \textbf{A}_i^{-1} \in \text{GL}_m(q) \gets \check{\textbf{A}}_i, \check{\textbf{A}}_i^{-1}$
    \State $\textbf{B}_i, \textbf{B}_i^{-1} \in \text{GL}_n(q) \gets \check{\textbf{B}}_i, \check{\textbf{B}}_i^{-1}$
    % Compute Gi
    \State $\textbf{G}_i \in \mathds{F}_q^{k \times mn} \gets \pi_{\textbf{A}_i, \textbf{B}_i}(\textbf{G}_0)$
    % Compute Compute T_i^-1 as a k*k submatrix of G_i
    \State $\textbf{T}_i^{-1} \in \mathds{F}_q^{k \times k} \gets \textbf{G}_i[;0,k-1]$
    % Convert Gi to systematic form
    \State $\textbf{G}_i \in \mathds{F}_q^{k \times mn} \cup \{\bot\} \gets \text{SF}(\textbf{G}_i)$
    % Retry if the matrix is not in systematic form
    \If{$\textbf{G}_i = \bot$}
        \State \textbf{goto} line 5
    \EndIf
    \EndFor
% Compute the pk from the data
\State $\text{pk} \in \mathcal{B}^{\ell_\textbf{pk}} \gets (\sigma_{\textbf{G}_0}~|~\text{CompressG}(\textbf{G}_1)~|~\ldots~|~\text{CompressG}(\textbf{G}_{s-1}))$
% Compute the sk from the data
\State $\text{sk} \in \mathcal{B}^{\ell_\textbf{sk}} \gets (\delta~|~\sigma_{\textbf{G}_0}~|~\text{Compress}(\textbf{A}_1^{-1})~|~\ldots~|~\text{Compress}(\textbf{A}_{s-1}^{-1})$\\
$\quad\quad\quad\quad\quad\quad\quad\quad\quad~|~\text{Compress}(\textbf{B}_1^{-1})~|~\ldots~|~\text{Compress}(\textbf{B}_{s-1}^{-1})$\\
$\quad\quad\quad\quad\quad\quad\quad\quad\quad~|~\text{Compress}(\textbf{T}_1^{-1})~|~\ldots~|~\text{Compress}(\textbf{T}_{s-1}^{-1}))$
% Return the public and secret key
\State \textbf{return} $\text{pk}, \text{sk}$
\end{algorithmic}
\end{algorithm}

\newpage

\section{Signing}
\begin{algorithm}
\caption{MEDS.Sign()}\label{alg:medssign}
\hspace*{\algorithmicindent} \textbf{Input:} secret key $\textbf{sk} \in \mathcal{B}^{\ell_\textbf{sk}}$, message $m \in \mathcal{B}^{\ell_m}$\\
\hspace*{\algorithmicindent} \textbf{Output:} signed message $m_s \in \mathcal{B}^{\ell_\text{sig} + \ell_m}$
\begin{algorithmic}[1]
% Initialize parsing index
\State $f_\text{sk} \gets \ell_\text{sec\_seed}$
% Parse sigma_G_0 from the secret key
\State $\sigma_{\textbf{G}_0} \gets \text{pk}[f_\text{sk}, f_\text{sk} + \ell_\text{pub\_seed} - 1]$
% Increment index
\State $f_\text{sk} \gets f_\text{sk} + \ell_\text{pub\_seed}$
% Construct G0
\State $\textbf{G}_0 \in \mathds{F}_q^{k \times mn} \gets \text{ExpandSysMat}(\sigma_{\textbf{G}_0})$
% Obtain all A_i from the secret key
\ForAll{$i \in \{1, \ldots, s - 1\}$}
    % Parse A_i from the secret key
    \State $\textbf{A}_i^{-1} \in \mathds{F}_q^{m \times m} \gets \text{Decompress}(\text{sk}[f_\text{sk}, f_\text{sk} + \ell_{\mathds{F}_q^{m \times m}}])$
    % Update the parsing index
    \State $f_\text{sk} \gets f_\text{sk} + \ell_{\mathds{F}_q^{m \times m}}$
\EndFor
% Obtain all B_i from the secret key
\ForAll{$i \in \{1, \ldots, s - 1\}$}
    % Parse B_i from the secret key
    \State $\textbf{B}_i^{-1} \in \mathds{F}_q^{n \times n} \gets \text{Decompress}(\text{sk}[f_\text{sk}, f_\text{sk} + \ell_{\mathds{F}_q^{n \times n}}])$
    % Update the parsing index
    \State $f_\text{sk} \gets f_\text{sk} + \ell_{\mathds{F}_q^{n \times n}}$
\EndFor
% Obtain all T_i from the secret key
\ForAll{$i \in \{1, \ldots, s - 1\}$}
    % Parse T_i from the secret key
    \State $\textbf{T}_i^{-1} \in \mathds{F}_q^{k \times k} \gets \text{Decompress}(\text{sk}[f_\text{sk}, f_\text{sk} + \ell_{\mathds{F}_q^{k \times k}}])$
    % Update the parsing index
    \State $f_\text{sk} \gets f_\text{sk} + \ell_{\mathds{F}_q^{k \times k}}$
\EndFor
% Generate a random seed
\State $\delta \in \mathcal{B}^{\ell_\text{sec\_seed}} \gets \text{Randombytes}(\ell_\text{sec\_seed})$
% Generate a random tree seed and salt from the secret seed
\State $\rho \in \mathcal{B}^{\ell_\text{tree\_seed}}, \alpha \in \mathcal{B}^{\ell_\text{salt}} \gets \text{XOF}(\delta, \ell_\text{tree\_seed}, \ell_\text{salt})$
% Construct t commitment seeds from the tree seed
\State $\sigma_0, \ldots, \sigma_{t-1} \in \mathcal{B}^{\ell_\text{tree\_seed}} \gets \text{SeedTree}_t(\rho, \alpha)$
% Generate t commitments from the challenge seeds
\ForAll{$i \in \{0, \ldots, t - 1\}$}
    % Construct a commitment seed for the current commitment
    \State $\sigma'_i \in \mathcal{B}^{\ell_\text{salt} + \ell_\text{tree\_seed} + 4} \gets (\alpha~|~\sigma_i~|~\text{ToBytes}(2^{1 + \lceil \log_2(t) \rceil + i, 4}))$
    % Generate seeds based on the current commitment seed
    \State $\sigma_{\tilde{\textbf{M}}_i} \in \mathcal{B}^{\ell_\text{pub\_seed}}, \sigma_i \in \mathcal{B}^{\ell_\text{tree\_seed}} \gets \text{XOF}(\sigma'_i, \ell_\text{pub\_seed}, \ell_\text{tree\_seed})$
    % Generate matrix ~M_i <- c0 and c1 represent the linear combination of codewords
    \State $\tilde{\textbf{M}}_i \in \mathds{F}_q^{2 \times k} \gets \text{ExpandSysMat}(\sigma_{\tilde{\textbf{M}}_i})$
    % Compute C = ~M_i * G_0 <- C contains the two codewords C0 and C1
    \State $\textbf{C} \in \mathds{F}_q^{2 \times mn} \gets \tilde{\textbf{M}}_i \textbf{G}_0$
    % Solve the system of equations to obtain A and B
    \State $\widetilde{\textbf{A}}_i \in \mathds{F}_q^{m \times m} \cup \{\bot\}, \widetilde{\textbf{B}}_i \in \mathds{F}_q^{n \times n} \cup \{\bot\} \gets \text{Solve}(\textbf{C})$
    % Retry if there was no solution
    \If{$(\widetilde{\textbf{A}}_i = \bot \textbf{ and } \widetilde{\textbf{B}}_i = \bot) \textbf{ or } \widetilde{\textbf{A}}_i \notin \text{GL}_m(q) \textbf{ or } \widetilde{\textbf{B}}_i \notin \text{GL}_n(q)$}
        \State \textbf{goto} line 18
    \EndIf
    % Compute G_i
    \State $\tilde{\textbf{G}}_i \in \mathds{F}_q^{k \times mn} \gets \pi_{\widetilde{\textbf{A}}_i, \widetilde{\textbf{B}}_i}(\textbf{G}_0)$
    % Convert G_i to systematic form
    \State $\tilde{\textbf{G}}_i \in \mathds{F}_q^{k \times mn} \cup \{\bot\} \gets \text{SF}(\tilde{\textbf{G}}_i)$
    % Retry if the matrix is not in systematic form
    \If{$\tilde{\textbf{G}}_i = \bot$}
        \State \textbf{goto} line 18
    \EndIf
\EndFor
% Create hash
\State $d \in \mathcal{B}^{\ell_\text{digest}} \gets \text{H}(\text{Compress}(\tilde{\textbf{G}}_0[;k,mn-1])~|~\ldots$\\
$\quad\quad\quad\quad\quad\quad~~|~\text{Compress}(\tilde{\textbf{G}}_{t-1}[;k,mn-1])~|~m)$
% Parse challenges from the hash
\State $h_0, \ldots, h_{t-1} \in \{0, \ldots, s-1\} \gets \text{ParseHash}_{s,t,w}(d)$
% For each challenge, compute the response
\ForAll{$i \in \{0, \ldots, t - 1\}$}
    % Only for non-zero challenges
    \If{$h_i \geq 0$}
        % Compute response
        \State $\kappa_i \in \mathds{F}_q^{2 \times k} \gets \tilde{\textbf{M}}_i T_{h_i}^{-1}$
    \EndIf
\EndFor
% Construct seed tree paths
\State $p \in \mathcal{B}^{\ell_\text{path}} \gets \text{SeedTreeToPath}_t(h_0, \ldots, h_{t-1}, \rho, \alpha)$
% Return the signature
\State \textbf{return} $m_s \in \mathcal{B}^{w \cdot \ell_{\mathds{F}_q^{2 \times k}} + \ell_\text{path} + \ell_\text{digest} + \ell_\text{salt} + \ell_\text{m} = \ell_\text{sig} + \ell_\text{m}}$\\
$\quad\quad\quad\quad= (\kappa_0~|~\ldots~|~\kappa_{t-1}~|~p~|~d~|~\alpha~|~m)$
\end{algorithmic}
\end{algorithm}

\end{document}